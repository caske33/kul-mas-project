\documentclass[10pt,a4paper]{article}

\usepackage[english]{babel}
\usepackage[utf8]{inputenc}
\usepackage{amsmath}
\usepackage{graphicx}
\usepackage{todonotes}
\usepackage{url}

\title{Extensions on Contract-Net Protocol for AGVs \\ \normalsize Multi-agent systems}

\author{Jens Claes \and Victor Le Pochat}

\date{June 3, 2016}

\newcommand{\proposalY}[1]{\todo[inline, color=green]{Proposal (by Yens): #1}}
\newcommand{\proposalV}[1]{\todo[inline, color=green]{Proposal (by Victor): #1}}

\newcommand{\commentY}[1]{\todo[inline, color=yellow]{Yens: "#1"}}
\newcommand{\commentV}[1]{\todo[inline, color=yellow]{Victor: "#1"}}

\newcommand{\taskY}[1]{\todo[inline, color=red]{@Yens, fix: #1}}
\newcommand{\taskV}[1]{\todo[inline, color=red]{@Victor, fix: #1}}

\newcommand{\outline}[1]{\todo[inline, caption={}, color=cyan]{\emph{Outline of what should come here}: #1}}

\begin{document}
\maketitle

\section{Introduction}
\commentV{Zie https://bitbucket.org/VictorLP/mas/wiki/Notities}
\subsection{Objectives}
\outline{\proposalV{Investigate the performance changes between DynCNET/confirmation and CNET. (bijbehorende question: what is the relative performance of cnet, confirmation and dyncnet)}}
The objective of this paper is to compare 2 extensions of Contract Net (CNET), namely Contract Net with Confirmation Protocol (CNCP) and Dynamic Contract Net (DynCNET) to CNET itself. We do this in a drone delivery setting where all agents (both the Drones and the Clients) act as homo economicus. We also investigate whether any of these protocols are suited for a Drone company to deliver packages from warehouses (with infinite stock) to clients. We will also check the influence of the number of Drones, Clients and Warehouses on the profit the company can make.
\subsection{Hypotheses}
\noindent\textit{Is there a difference between CNET, CNCP and DynCNET in the profit the company can make?} \\
We expect that the company can make more profit using DynCNET than using CNCP than using CNET. \\

\noindent\textit{Is there a difference in average delivery time between CNET, CNCP and DynCNET?} \\
We expect that CNCP and CNET will deliver faster than DynCNET (as DynCNET has wasted time when it switches). We don't expect a difference between CNCP and CNET as both will start delivering fast and won't waste any time by switching.\\

\noindent\textit{Is there a difference in the number of clients that are not delivered between CNET, CNCP and DynCNET?} \\
We expect that CNET and CNCP will deliver the same amount of clients and DynCNET less. Again because of the switching.\\

\noindent\textit{Is there a difference in the number of messages that needs to be exchanged between CNET, CNCP and DynCNET?}\\
We expect that CNCP will need less messages because we expect less rounds of bidding. CNET will take more messages but still less than DynCNET which needs a lot of messages to make sure the best contract is being executed. \\

\noindent\textit{Is it possible to make a profit using any of the 3 approaches?}\\
We expect that all 3 approaches should be profitable give the right number of warehouses, drones and clients.\\

\noindent\textit{What is the influence of the number of drones on the profit?}\\
We expect the profit to scale linearly with the number of drones. Once all clients start to get delivered, saturation will set in.\\

\noindent\textit{What is the influence of the number of warehouses on the profit?}\\
We expect the profit to scale with the number of warehouses. As more warehouses become available, it will become more likely to have a warehouse with a low price for the product near the customer.\\

\noindent\textit{What is the influence of the number of clients on the profit?}\\
We expect the profit to rise until the fines of not delivering start to become too large.\\

\outline{In profit: DynCNET better than Confirmation better than CNET \\ In messages: CNET better than confirmation better than DynCNET}

\section{Theory}
\outline{Theoretische uiteenzetting van wat in de drie relevante papers staat}
\commentY{\cite{DynCNET}, \cite{CNET}, \cite{CNCP}, \cite{CNETStandard}}

\section{Multi-agent system design}
\outline{Wat wij concreet doen}
\subsection{Design}
\outline{\begin{itemize}
        \item drones leveren orders uit warehouses aan clients (1 order per client)
        \item bid berekening (statistisch relevant)
        \item messaging gedetailleerd bespreken of niet?
    \end{itemize}}
\subsection{Comparison with existing MAS from literature}
\outline{
    \begin{itemize}
        \item warehouses oneindige capaciteit
        \item tasks uit literature zijn orders bij ons
        \item verschillen in messaging
        \item concretere berekening bid
        \item hier de state machines? (of eerder bij design)
    \end{itemize}
    }
\commentY{\cite{TentativeBidding}, \cite{LeveledCommitment}, \cite{CNETIterativeStandard}}

\section{Experiments}
\subsection{Setup}
\outline{\begin{itemize}
        \item independent: aantal drones, warehouses, clients
        \item dependent: profit, aantal clients delivered, aantal messages, delivery time, ... (\texttt{ExperimentResult})
        \item other: battery drain, kost pakketjes, oplaadkost, ... (\texttt{Variables.kt} + \texttt{PackageType.kt})
    \end{itemize}}
\subsection{Results}
\subsection{Analysis}

\section{Conclusion}
\bibliographystyle{plain}
\bibliography{report}

\end{document}